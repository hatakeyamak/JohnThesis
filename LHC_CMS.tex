
\section{The Large Hadron Collider}

%Talk about basics of LHC circumfrance, energies, location... and other basic stuff to intro before this paragraph which kind of does staright into details.

The LHC is located at CERN near Geneva, Switzerland. It has a circular shape and is 27 kilometers in circumference. It is about 100 meters underground and as shown in ~\ref{fig:LHC} it simply goes right under the towns and farmland in the area. To look for undiscovered particles the LHC collides protons at really high energies. After the recent upgrade the LHC now collides protons with a center of mass energy of 13 TeV, meaning the individual protons each have 6.5 TeV of energy. The theory of special relativity says that a particle can only approach the speed of light. As the protons in the LHC are traveling close to the speed of light it is more useful to use their energy rather than their speed. In addition, mass is often put into units of energy of the speed of light squared making the mass to energy conversion simple. 

\begin{figure}
\centering
\includegraphics[width=0.8\linewidth]{Figures/LHC.png}
\caption{A picture of the LHC (outlined in yellow) with the different detectors highlighted}
\label{fig:LHC}
\end{figure}

The process the Large Hadron Collider (LHC) uses sounds simple when put into common terms. It accelerated protons to speeds very close to the speed of light and collides them in the center of a detector to see what comes out of these collisions. However actually doing this is not simple at all. To start this entire process electrons are stripped off of hydrogen gas supplying the protons for the LHC. After this the protons are put into a series of different accelerator each designed to accelerator to protons to higher and higher energies. The first accelerator is the Linac2 linear accelerator, which gets the protons up to 50 MeV. Then they are sent to the Proton Synchrotron Booster which can push them to 1.4 GeV, after which the Proton Synchrotron ring accelerates the protons to 25 GeV. At this point the protons are sorted to control the frequency at which the collisions will occur. The protons are sorted into bunches such that a proton bunch passes by every 25ns. Each bunch has about 100 billion protons in it. After sorted into these bunches the protons are sent to the Super Proton Synchrotron where they achieve an energy of 450 GeV. Finally, the protons are fed into the LHC where they will be accelerated to their highest energy. An illustration of all of these accelerators is shown in figure~\ref{fig:acceleratorcomplex}.

\begin{figure}
\centering
\includegraphics[width=0.8\linewidth]{Figures/acceleratorcomplex.jpg}
\caption{Chain of accelerators feeding into the LHC}
\label{fig:acceleratorcomplex}
\end{figure}

When the LHC has accelerated its particles to its peak energy it collides the particles in the middle of four different detectors. LHCb, ALICE, ATLAS, and CMS are the names of the four detectors that were built to study collisions supplied by the LHC. The CMS detector is the one I worked on during my undergraduate career, and is discussed in detail below.

\section{The CMS Detector}
The CMS detector is designed to detector particles coming out of the proton-proton collisions supplied by the LHC~\cite{CMS}. As many different types of particles come out of these collisions the CMS detector is split up into several different sub-detectors. Going from the closest to the collision point outward, first is the silicon tracker. This is responsible for measuring the total energy of the particles in the detector which is important for reconstructing the events using conservation of energy. Next is the Electromagnetic Calorimeter which is responsible for detecting a myriad of charged particles such as photon, and electrons. Then there is the Hadron Calorimeter which is responsible for detector hadrons, which are particles made up of quarks, such as protons neutrons and pions. These calorimeters work by covering the possible directions out of the collision with scintillator tiles. When the particles hit these tiles they lose some of their energy depositing light proportional to the energy of the particle. The particle will eventually stop having lost all of its energy giving us a reading on its total energy. Finally, are the muon chambers. Muons are interesting particles because while they are heavy particles that will decay they have long enough lifetimes to make it through the detector. In addition, they tend to not be stopped by the detector like the two prior calorimeters almost always stop the particles they detect. Since there is a magnetic field encompassing the detector and the muons have a charge they will curve as they pass through the muon chambers. While not completely stopping the muons will give a reading in the chambers showing its path of travel. By measuring the curve of this path of travel we can measure the momentum. By combining the the data from all the different sub detectors we can get a picture of almost all of the particles that came out of the collision. There are exceptions such as neutrinos which tend to pass through anything leaving no trace. Nevertheless we are able to use this to recreate what exactly came out of the collision, which allows us to find evidence of particles like the top quark which decays so quickly it does not even make it to the detector.

The CMS detector is designed to detector particles coming out of the proton-proton collisions supplied by the LHC. As many different types of particles come out of these collisions the CMS detector is split up into several different sub-detectors. Going from the closest to the collision point outward, first is the Silicon Tracker. The Silicon Tracker is capable of finely measuring the path of the particles that go through it. The CMS detector produces a 4 Tesla magnetic field throughout most of the detector. As charged particles move through the magnetic field, their path curves. The Silicon Tracker allows us to measure the curvature of their path and find their momentum. Next is the Electromagnetic Calorimeter which is mainly responsible for measuring the energy of photons and electrons that come out of the collision. Then there is the Hadron Calorimeter (HCAL) which is responsible for detector hadrons, which are particles made up of quarks, such as protons neutrons and charges pions. The HCAL works but covering almost all $4\pi$ direction out of the collision with scintillator tile. When particles go through the scintillator tile they interact with the molecules exciting them while the particles lose some of their energies. The molecules in the scintillator will then go back to their original energy state emitting a photon. These photons are gather by an optical fiber. By counting these photons we can measure the energy lost by the particle. The detector has enough scintillator tile in the path of the particle to make it highly probably the particle will lose all of its energy in the detector. While this is still a simplification of the process counting the number of photons from the scintillator tile is the basis of the measurement of the energy of the particle. Finally, are the Muon Chambers. Unlike other particles the muon tends to just go through the scintillator only losing a small portion of its energy. This means the muons will reach the last section of the detector mostly unperturbed. Since the muons are charges their path will be curved by the magnetic field so their momentum is measured by a process similar to that of the Silicon Tracker. An illustration of the of the different sub-detectors and the particles that will interact with them is shown in figure~\ref{fig:CMSlayout}.

Even with all of these sub-detector there are still particles not directly detected. Neutrinos for example tend to go through everything without interacting so they do not appear in our detectors signals. Several of the particles on table~\ref{tab:particles} decay before even reaching the detector. To find evidence of these particles we have to find indirect evidence of them. The top quark for example nearly 100\% of the time decays into a bottom quark and a W boson. The W boson then either decays into a charged lepton neutrino pair or a light quark anti-quark pair. By finding evidence of these particles in close proximity we can deduce these came from a top quark.

\begin{figure}
\centering
\includegraphics[width=\linewidth]{Figures/CMSlayout.jpg}
\caption{A slice of the CMS detector highlighting the different sub-detectors and showing different particles and where they are stopped}
\label{fig:CMSlayout}
\end{figure}

\subsection{Hadron Calorimeter}
One of the main jobs of the calorimeter is to measure the energy of the of the incident particle. When the particle hits the scintillator tile and loses energy it emits photons into the tile proportional to the amount of energy it lost. In order to collect these photons a optical fiber is put in the outer edge of the tile. A picture of the scintillator tile is shown in figure~\ref{fig:Tile}. This fiber will capture the photons and channel them to a photo detector, which will take the light signals and convert them to charge signals. The original photo detector for the HCAL was the hybrid photo-diodes (HPD) but they are being replaced by silicon photomultipliers as apart of some of the upgrades on the CMS detector. 

\begin{figure}
\centering
\includegraphics[width=0.6\linewidth]{Figures/Tile.png}
\caption{A picture of a scintillator tile with the optical fiber around the edges.}
\label{fig:Tile}
\end{figure}


There are three main section to the HCAL. There is the Hadron Barrel (HB)~\cite{HB}, which surrounds the beam-line like the edge of a cylinder, the Hadron End-cap (HE), which caps off the cylinder made by HB. These two parts make a cylinder which has the beam-line going through the center and the collision point in the center. Lastly there is the Hadron Front-cap (HF) which is shaped like HE but is located much further away from the collision point~\cite{HF}. To describe the geometry of the detector we use a coordinate system related to spherical coordinates with the origin being the collision point and polar angle of 0 being along the beam-line. Phi is the azimuthal angle and Eta is related to the polar angle by $\eta = -\ln(\tan(\theta/2))$ which gives a value of 0 perpendicular to the beam-line and $\pm\infty$ along the beam-line. Usually these coordinates are arranged into discrete integers values denoted iphi and ieta, which are arranged such that a scintillator tile covers one ieta and one iphi as shown in figure~\ref{fig:Depth} but there are exception to this. For the distance from the beam-line depth is used which is also a discrete value but there are often more than one scintillator tiles in a single depth. 


\begin{figure}
\centering
\includegraphics[width=\linewidth]{Figures/Depthsegmentation.pdf}
\caption{Layout of the HCAL showing a single iphi slice of HE HB and HF. Each box represents a scintillator tile}
\label{fig:Depth}
\end{figure}

\subsection{Silicon Photomultipliers}
The SiPM does its job of counting the photons from the scintillator by shining them on a pixel face. This is just a circular array of really small pixels. A picture of the SiPM is in figure~\ref{fig:SiPM} showing several pixel faces. There are about 33000 pixels on a 3.3 mm diameter pixels face. When an individual photon hits one of the pixels on the pixel face it causes a capacitor to fire off a particular amount of charge. The total output charge of the SiPM is the sum of all of the pixels output charge. Theoretically one just needs to sum up the total amount of output charge of the SiPM and one can count the number of incident photons since each pixel will fire off a set amount of charge. The counting of the number of incident photons will give a measurement of the incident particles energy. There are some things that make the SiPMs reading not so simple. SiPMs are non-linear devices meaning the total output charge of the SiPM does not necessarily increase linearly with the number of incident photons. In fact, experiments have shown that graphing the number of incident photons vs the SiPM output charge it looks more like a square root function rather than a straight line. At low incident photon count on the order of 1000 photons the SiPM is very close to a linear device but as the incident number of photons increases the output charge does not increase at the same rate. There are several factors that contribute to SiPM non-linearity but the two main ones are cross talk and saturation. Cross talk is when a pixel is activated by a photon there is a chance that this activated pixel could activate some of its neighboring pixels artificially increase the output charge of the SiPM. When the photon hits a pixel it excites an electron causing an electron cascade and charge to flow. This excited electron has a chance to emit an IR photon which could go in any direction. If it hits one of the neighboring pixels it could activate it just like an incident photon. 

Saturation is when a pixel is hit in rapid succession by multiple photons. When a pixel is hit by a photon it discharges its capacitor which is the source of its output charge. After the pixel fires it needs some time on the order of 10 nanoseconds to recharge its capacitor. If the pixel is hit it simply fires off whatever charge it has in its capacitor at the time which could be reduced from its normal charge. Given that the SiPM has about 33000 pixels on its pixel face, when the number of incident is on the order of 1000 this effect is insignificant, but particles can produce much more photons where this effect can be much more significant. The non-linear properties of the SiPM makes it a bit harder to get accurate measurements, but there are ways to study them which will allow us to compensate for these affects.

\begin{figure}
\centering
\includegraphics[width=\linewidth]{Figures/SiPM.jpg}
\caption{A picture of a Silicon Photomultiplier}
\label{fig:SiPM}
\end{figure}




