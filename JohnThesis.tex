
\documentclass
[]
{thesis}

\usepackage{graphicx}
\usepackage{amsmath}
\usepackage{lipsum} 
\usepackage{hyperref}  
\usepackage{bookmark}

\title{Title Filler}
\titleCaps{TITLE FILLER}
\author{John Lawrence}
%\holding{}
%\seeking{}
%\degree{}
\date{May 2018}

% Administration
%\graduateDean{}
\department{Department of Physics}
%\departmentChair{}

% Supervisor (adviser / mentor)
\supervisor{Kenichi Hatakeyama}
\supervisorTitle{Director}

% Honors Program Director
\honorsProgramDirector{Elizabeth Corey}

% Abstract
\abstract{Yet to be filled in.}


\begin{document}
\pdfbookmark[-1]{CONTENT}{bookmark:Content}
	
\chapter{Basic Particle Physics and CERN}
\label{chap:one}
	
Ever since ancient Greece people have been trying to discover what makes up everything. Back then they said that everything was comprised of four elements: earth, fire, water, and air. Although they use a few more than four elements, scientists today are still trying to categorize the most fundamental element that makes up everything. Democritus, an ancient Greek scientist, proposed that everything if divided enough times could be broken down into atoms their most fundamental object. Although some people were a little ambitious and named something the atom that was not the most fundamental substance scientists are still searching for the atom that Democritus theorized. 

In the 19th century John Dalton believed he found the most fundamental particle and named it the atom, but as time passed there was very strong evidence that even atoms had internal components and structure. This was proven when J. J. Thompson in the late 19th century found the electron. This was not only the first fundamental particle discovered, but also the beginning of particle physics as a field of study. While it was an insightful discovery that led to many inventions, there were still many unknown things about the atom. About a decade after Thompson, Robert Millikan finely measured the charge and mass of the electron, but this raised many questions about the atom. It was not until Ernest Rutherford in the early 20th century showed that the atom is mostly empty space with a very dense core called the nucleus which has an opposite charge to that of the electron.

These discoveries showed there were more things to explore past the atom but in parallel there were equally important theories being made. Quantum mechanics is essential to the study of sub-atomic particles as they no longer behave according to classical mechanics. During this time Max Planck was doing his work in the quantization of energy. This set the ground work for quantum mechanics and also served as a basis to Einstein's work. In 1905 Einstein formulated his theory of special relativity, which had several major impacts on the field of particle physics. One the major ones was quantization of light into particles called photons. Although light had long been thought of as a wave Einstein showed many of its particle like properties leading to the wave-particle duality of light. Eventually De Brogile extended this duality to all particles theorizing that all particles including electrons and other sub-atomic particles could be described using waves or particles.

Einstein's theory of relativity had another major impact on particle physics. Probably the most famous of Einstein's equation, $E = mc^2$ shows the mass energy relationship. One of the implications of quantum mechanics is that if something can happen it will happen under a certain probability. This means that given the right conditions and enough energy one can simply create mass. There are other conservation laws that need to be followed that are still relevant to particle physics, but conservation of energy is one of the biggest limiters as it says to create more massive particles more energy is needed. 

As these theories became more complete, they allowed for the further description of sub-atomic particles, like the states of the electrons in the atoms and the protons and neutrons that make up the nucleus of the atom. Scientist also began to find evidence of fundamental particles not in atoms. In 1932 Carl Anderson found evidence of a Muon from cosmic rays. In addition, as scientists began to probe higher energies they found evidence of internal structure within protons and neutrons. In 1968 at the Standford Linear Accelerator Center they found evidence of particles within protons and neutrons, which were later called quarks.

The discovery of more and more fundamental particles led to the formulation of the standard model, which is like the periodic table of elements for elementary particles. The standard model has three main groups, the first of which is the quarks. There are six different flavors of quarks which are separated into three generations. Quarks have four distinct properties, they have half integer intrinsic spin, and they are affected by the electromagnetic, weak, and strong force. Each generation of quarks has one with a +2/3 charge and one with a -1/3 charge. For the strong force, each quark has a charge like property called color. There are also six respective anti quarks which have the same properties, but have opposite charge and strong force color. One thing that makes quarks interesting is that to date they have not been found in isolation for any significant amount of time. Quarks are usually found in groups of two or three, though more are theoretically possible. Protons and neutrons are examples of baryons, groups of three quarks. There can also be particles called mesons, which are comprised of a quark and anti quark. 

\begin{center}
\begin{tabular}{l c r}
	First Generation & Second Generation & Third Generation \\
	Up & Charm & Top \\
	Down & Strange & Bottom \\
\end{tabular}
\end{center}

The next group in the standard model is the leptons. Like quarks, leptons have half integer spin and are affected by the weak force, but are not affected by the strong force and some are not even governed by the electromagnetic force. There are also three generations of leptons each containing a flavor along with its respective neutrino particle. Non-neutrino leptons have a charge of -1 and their respective anti particles have a charge of +1. Neutrinos are a bit odd in that they have no charge and are near mass-less. Because of this neutrinos barely interact with any particles, and there are many neutrinos passing through the earth every second without having any affect.

\begin{center}
\begin{tabular}{l c r}
	First Generation & Second Generation & Third Generation \\
	Electron & Muon & Tau \\
	Electron Neutrino & Muon Neutrino & Tau Neutrino \\
\end{tabular}
\end{center}

Finally, there are the bosons. Bosons are different from both leptons and quarks in that they have integer spin, but the individual bosons differ in which fundamental force affects them. While leptons and quarks combine together to form complex arrangements such as atoms and all of matter, bosons serve as the force carriers. Whenever there is a interaction through a fundamental force there is a boson that mediates that interaction, although gravity has yet to be explained using the standard model.
 
\begin{center}
\begin{tabular}{l r}
	Z boson & Weak Force Carrier \\
	W boson & Weak Force Carrier \\
	Photon & Electromagnetic Force Carrier \\
	Gluon & Strong Force Carrier \\
	Higgs Boson & Gives Particles Mass \\
\end{tabular}
\end{center}

The standard model essentially serves as a list of known fundamental particles but the road to completing it was not an easy one. The electron is the only particle on the standard model that is found in isolation at energies used on earth. To find the next particle, the muon, scientist had to look at cosmic rays, which provided the energy needed to create these particles. While neutrinos are plentiful they interact so infrequently scientist are still struggling to create an efficient detector for them. To discover the remaining particles on the standard model they had to use particle colliders. Throughout the last few decades, scientists have built colliders to probe higher and higher energies, to make them capable of discovering more fundamental particles. In 2012, using the Large Hadron collider at CERN scientists were able to find evidence of the Higgs boson, the final particle on the standard model. However, there are still phenomenon in the universe that cannot be explained using the standard, such as gravity and dark matter. There are several theories beyond the standard model that offer explanations for these, but they also say there are more fundamental particles to be discovered at higher energies. This is one of the main goals of the CERN collaboration. 

Fermions and quarks are the building blocks of all of matter. All of the atoms on the periodic table are made up of electronics, and up and down quarks, which make up protons and neutrons. Aside from the neutrinos, the particles in higher generations are extremely similar to the respective particle in the first generation except for being a substantially higher mass. Because of this these particles have very short lifetimes and are only found in high energy interactions. For instance, when cosmic rays hit the earths atmosphere they create muons. To find the particles like the top quark, which is over 1000 times more massive than the muon, the colliders need to collide particles together at even higher energies. The Large Hadron Collider at CERN, is the highest energy collider humans have built to date and has started searching for particles at energies previously not achieved.

\chapter{Large Hadron Collider and Compact Muon Solenoid}
\label{chap:two}


\end{document}
