
\documentclass
[]
{thesis}

\usepackage{graphicx}
\usepackage{amsmath}
\usepackage{lipsum} 
\usepackage{hyperref}  
\usepackage{bookmark}

\title{Title Filler}
\titleCaps{TITLE FILLER}
\author{John Lawrence}
%\holding{}
%\seeking{}
%\degree{}
\date{May 2018}

% Administration
%\graduateDean{}
\department{Department of Physics}
%\departmentChair{}

% Supervisor (adviser / mentor)
\supervisor{Kenichi Hatakeyama}
\supervisorTitle{Director}

% Honors Program Director
\honorsProgramDirector{Elizabeth Corey}

% Abstract
\abstract{Yet to be filled in.}


\begin{document}
\pdfbookmark[-1]{CONTENT}{bookmark:Content}
	
\chapter{Basic Particle Physics and CERN}
\label{chap:one}
	
Ever since ancient Greece people have been trying to discover what makes up everything. Back then they said that everything was comprised of four elements: earth, fire, water, and air. Although they use a few more than four elements, scientists today are still trying to categorize the most fundamental element that makes up everything. Democritus, an ancient Greek scientist, proposed that everything if divided enough times could be broken down into atoms their most fundamental object. Although some people were a little ambitious and named something the atom that was not the most fundamental substance scientists are still searching for the atom that Democritus theorized. 

In the 19th century John Dalton believed he found the most fundamental particle and named it the atom, but as time passed there was very strong evidence that even atoms had internal components and structure. This was proven when J. J. Thompson in the late 19th century found the electron, which ended up being a component of the atom. This was not only the first fundamental particle discovered, but also the beginning of particle physics as a field of study. While it was an insightful discovery that led to many inventions, there were still many unknown things about the atom. About a decade after Thompson, Robert Millikan finely measured the charge and mass of the electron, but this raised many questions about the atom. It was not until Ernest Rutherford in the early 20th century showed that the atom is mostly empty space with a very dense core called the nucleus which has an opposite charge to that of the electron. He did this by shooting alpha particles, made up of two protons and two neutrons, at gold foil and observing their deflection angle. In this experiment most of the alpha particles passed through unperturbed while a handful were majorly deflected. From this Rutherford made the conclusion most of the atom was empty space while having a highly dense core called the nucleus. 

These discoveries showed there were more things to explore past the atom but in parallel there were equally important theories being made. Quantum mechanics is essential to the study of sub-atomic particles as they no longer behave according to classical mechanics. During this time Max Planck was doing his work in the quantization of energy. This set the ground work for quantum mechanics and also served as a basis to Einstein's work. In 1905 Einstein formulated his theory of special relativity, which had several major impacts on the field of particle physics. One the major ones was quantization of light into particles called photons. Although light had long been thought of as a wave Einstein showed many of its particle like properties leading to the wave-particle duality of light. Eventually De Brogile extended this duality to all particles theorizing that all particles including electrons and other sub-atomic particles could be described using waves or particles.

Einstein's theory of relativity had another major impact on particle physics. Probably the most famous of Einstein's equation, $E = mc^2$ shows the mass energy relationship. One of the implications of quantum mechanics is that if something can happen it will happen under a certain probability. This means that given the right conditions and enough energy one can simply create mass. There are other conservation laws that need to be followed that are still relevant to particle physics, but conservation of energy is one of the biggest limiters as it says to create more massive particles more energy is needed. 

As these theories became more complete, they allowed for the further description of sub-atomic particles, like the states of the electrons in the atoms and the protons and neutrons that make up the nucleus of the atom. Scientist also began to find evidence of fundamental particles not in atoms. In 1932 Carl Anderson found evidence of a Muon from cosmic rays. In addition, as scientists began to probe higher energies they found evidence of internal structure within protons and neutrons. In 1968 at the Standford Linear Accelerator Center they found evidence of particles within protons and neutrons, which were later called quarks.

The discovery of more and more fundamental particles led to the formulation of the standard model, which is like the periodic table of elements for elementary particles. The two main groups of particles on the standard model are fermions, which have half integer spin and bosons, which have integer spin. The main physical difference this causes is from the Pauli exclusion principle, which allows fermions to build into larger arrangements. Because of this fermions are the building blocks of the macroscopic world as they makeup protons neutrons, and atoms.   

A sub category of fermionic particles on the standard model are the quarks, which distinguish themselves from other fermions in their interaction via the strong force. Quarks are divided into three generation and each generation has one quark with a +2/3 charge and one with a -1/3 charge. In addition, quarks also have a color, either red, green, or blue, which serves as the "charge" for the strong force. For each quark there is also a corresponding anti-quark which has the same properties just opposite charge and color. Quarks do not remain in isolation for long, but instead are usually found in groups of two or three. A grouping of three quarks is bound by the strong force and called a baryon, the most common of which are protons and neutrons. The known arrangements of baryons are such that they have integer charge and neutral color. As such, in a baryon formation there is one red, one green, and one blue quark which adds up to a neutral color. This arrangement keeps the quarks tightly bound so it takes a large amount of energy to separate the individual quarks. There are also formations of a quark and anti-quark called mesons.  

\begin{center}
\begin{tabular}{l c r}
	First Generation & Second Generation & Third Generation \\
	Up & Charm & Top \\
	Down & Strange & Bottom \\
\end{tabular}
\end{center}

The next group of fermionic particles are the leptons. Like quarks, leptons have half integer spin and are affected by the weak force, but are not affected by the strong force and some are not even governed by the electromagnetic force. Because of this leptons do not form tightly bounded states like the quarks. There are also three generations of leptons each containing a flavor along with its respective neutrino particle. Non-neutrino leptons have a charge of -1 and their respective anti particles have a charge of +1. Neutrinos are a bit odd in that they have no charge and are near mass-less. Because of this neutrinos barely interact with any particles, and there are many neutrinos passing through the earth every second without having any affect.

\begin{center}
\begin{tabular}{l c r}
	First Generation & Second Generation & Third Generation \\
	Electron & Muon & Tau \\
	Electron Neutrino & Muon Neutrino & Tau Neutrino \\
\end{tabular}
\end{center}

Finally, there are the bosons. Bosons are different from both leptons and quarks in that they have integer spin, but the individual bosons differ in which fundamental force affects them. While leptons and quarks combine together to form complex arrangements such as atoms and all of matter, bosons serve as the force carriers. Whenever there is a interaction through a fundamental force there is a boson that mediates that interaction, although gravity has yet to be explained using the standard model. Although some are basically mass-less, others have masses thousands of times greater and the electron. 
 
\begin{center}
\begin{tabular}{l r}
	Z boson & Weak Force Carrier \\
	W boson & Weak Force Carrier \\
	Photon & Electromagnetic Force Carrier \\
	Gluon & Strong Force Carrier \\
	Higgs Boson & Gives Particles Mass \\
\end{tabular}
\end{center}

The standard model essentially serves as a list of known fundamental particles but the road to completing it was not an easy one. The electron is the only particle on the standard model that is found in isolation at energies used on earth. To find the next particle, the muon, scientist had to look at cosmic rays, which provided the energy needed to create these particles. While neutrinos are plentiful they interact so infrequently scientist are still struggling to create an efficient detector for them. To discover the remaining particles on the standard model they had to use particle colliders. Throughout the last few decades, scientists have built colliders to probe higher and higher energies, to make them capable of discovering more fundamental particles. In 2012, using the Large Hadron Collider at CERN scientists were able to find evidence of the Higgs boson, the final particle on the standard model. However, there are still phenomenon in the universe that cannot be explained using the standard, such as gravity and dark matter. There are several theories beyond the standard model that offer explanations for these, but they also say there are more fundamental particles to be discovered at higher energies. This is one of the main goals of the CERN collaboration. 

Fermions and quarks are the building blocks of all of matter. All of the atoms on the periodic table are made up of electronics, and up and down quarks, which make up protons and neutrons. Aside from the neutrinos, the particles in higher generations are extremely similar to the respective particle in the first generation except for being a substantially higher mass. Because of this these particles have very short lifetimes and are only found in high energy interactions. For instance, when cosmic rays hit the earths atmosphere they provide the energy to create muons. To find the particles like the top quark, which is over 1000 times more massive than the muon, the colliders need to collide particles together at even higher energies. The Large Hadron Collider at CERN, is the highest energy collider humans have built to date and has started searching for particles at energies previously not achieved.

\chapter{Large Hadron Collider and Compact Muon Solenoid}
\label{chap:two}

The process the Large Hadron Collider (LHC) uses sounds simple when put into common terms. It accelerated protons to speeds very close to the speed of light and collides them in the center of a detector to see what comes out of these collisions. However actually doing this is not simple at all. The theory of special relativity says that a particle can only approach the speed of light. As the protons in the LHC are traveling close to the speed of light it is more useful to use their energy rather than their speed. In addition, mass is often put into units of energy of the speed of light squared making the mass to energy conversion simple. To start this entire process electrons are stripped off of hydrogen gas supplying the protons for the LHC. After this the protons are put into a series of different accelerator each designed to accelerator to protons to higher and higher energies. The first accelerator is the Linac2 linear accelerator, which gets the protons up to 50 MeV. Then they are sent to the Proton Synchrotron Booster which can push them to 1.4 GeV, after which the Proton Synchrotron ring accelerates the protons to 25 GeV. At this point the protons are sorted to control the frequency at which the collisions will occur. The protons are sorted into bunches such that given a fixed point on the LHC a proton bunch passes by every 25ns. Each bunch has about 100 billion protons in it. After sorted into these bunches the protons are sent to the Super Proton Synchrotron where they achieve an energy of 450 GeV. Finally, the protons are fed into the LHC where they will be accelerated to their highest energy.



\end{document}
