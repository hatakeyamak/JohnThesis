
\documentclass
[]
{thesis}

\usepackage{graphicx}
\usepackage{amsmath}
\usepackage{lipsum} 
\usepackage{hyperref}  
\usepackage{bookmark}

\title{Title Filler}
\titleCaps{TITLE FILLER}
\author{John Lawrence}
%\holding{}
%\seeking{}
%\degree{}
\date{May 2017}

% Administration
%\graduateDean{}
\department{Department of Physics}
%\departmentChair{}

% Supervisor (adviser / mentor)
\supervisor{Kenichi Hatakeyama}
\supervisorTitle{Director}

% Honors Program Director
\honorsProgramDirector{Elizabeth Corey}

% Abstract
\abstract{Something should probably go here but I have yet to decide on a detailed outline so for now I will just babble for a bit.}


\begin{document}
\pdfbookmark[-1]{CONTENT}{bookmark:Content}
	
\chapter{Background: Basic Particle Physics and CERN}
\label{chap:one}
	
Ever since ancient Greece people have been trying to discover what makes up everything. Back then they said that everything was comprised of four elements: earth, fire, water, and air. Although they use a few more than four elements, scientists today are still trying to categorize the most fundamental element that makes up everything. Democritus, an ancient Greek scientist, proposed that everything if divided enough times could be broken down into atoms their most fundamental object. Although some people were a little ambitious and named something the atom that was not the most fundamental substance scientists are still searching for the atom that Democritus theorized. 

Starting in about the 19th century scientist started to discover that the atom could be further disassembled into even smaller components. These experiments led to the discovery of sub atomic particles like the proton neutron and electron. Now experiments are being conducted to probe the particles, and others on that level, that make up protons and neutrons. The experiments to probe lower scales have become drastically harder. When Rutherford made the discovers of the nucleus of the atom his experiments were basically setup and run by a single graduate student. The leading collaboration for this field of research CERN has a particle accelerator, the Large Hadron Collider or LHC, which is 27 kilometers in circumference and has four different detectors. One detector, the Compact Muon Solenoid or CMS, weighs over 30 million pounds and has a collaboration involving more than 3500 scientists, engineers, and students from 202 institutes from 47 different countries [1]. 

Scientists today are trying to classify the different particles that make up all matter. To that end they created the standard model. The standard model is essentially a listing of all known fundamental particles. First there are the hadrons. Hadrons are specifically classified as fermionic particles that have strong force interactions. These particles make up protons, neutrons, and a myriad of other particles.
\\
\begin{center}
\begin{tabular}{l c r}
	First Generation & Second Generation & Third Generation \\
	Up & Charm & Top \\
	Down & Strange & Bottom \\
\end{tabular}
\end{center}
Next are the leptons. Leptons are fermions like hadrons but do not interact through the strong force which means their interactions are dominated by the weak and electromagnetic forces.
\\
\begin{center}
\begin{tabular}{l c r}
	First Generation & Second Generation & Third Generation \\
	Electron & Muon & Tau \\
	Electron Neutrino & Muon Neutrino & Tau Neutrino \\
\end{tabular}
\end{center}
Finally, there are the bosons. Bosons are not fermions like hadrons and leptons and are responsible for the forces between particles.
\\
\begin{center}
\begin{tabular}{l r}
	Z boson & Weak Force Carrier \\
	W boson & Weak Force Carrier \\
	Photon & Electromagnetic Force Carrier \\
	Gluon & Strong Force Carrier \\
	Higgs Boson & Gives Particles Mass \\
\end{tabular}
\end{center}
The standard model essentially serves as a list of known fundamental particles but there are still many phenomenon that the standard model does not explain. Experiments being done today are exploring the mysterious of the known and unknown particles. 
\end{document}
