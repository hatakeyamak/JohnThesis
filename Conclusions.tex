The CMS experiment at the LHC has had many successes, including the discovery of the Higgs Boson in 2012. In order to search for undiscovered particles, the LHC was upgraded to be able to produce collisions at higher energies and at a higher rate. The CMS detector also had to undergo an upgrade in order to benefit fully from the upgraded LHC. I worked on the testing of new readout modules for this upgrade in the CMS detector. The readout modules performed as expected in the tests and showed they were capable of performing the task for which they were designed. 

In this thesis, I presented my analysis on the new readout modules, with a focus on SiPM features and pulse shape extraction. The pulse shape analysis using the test beam data was very successful. It was expected that the pulse shape would be similar to a Landau Gaussian function, which is what the analysis showed. The signal pulse shape was also found to be approximately the same over different charge ranges. This pulse shape will allow the extraction of individual particle signals in the CMS detector. 

The non-lienarity of the SiPM was another feature I studied, but it could not be analyzed in depth due to limitation of the signal amplitudes from the test beam. The SiPM is approximately linear at low incident photons. To study the SiPM nonlinearity, a particle at an energy that will produce a number of photons in the nonlinear range is needed, but the test beam was not capable of producing particles at the needed energies. The test beam did show an energy range for which the SiPM is nearly linear.

The SiPM simulation was constructed to supplement the other analyses and illustrate different features of the SiPM in an intuitive manner. When the results of the simulation were compared to results from other analyses, however, it showed that there were still some flaws in the construction of the simulation. The two main things I studied in my work were SiPM nonlinearity and the pulse shape. The pulse shape from the simulation was similar to the pulse shape from test beam data, but the simulation pulse shape had a narrower base. The nonlinearity correction curve from the simulation was much lower than other experimental measurements.

The reason for the discrepancy between the simulation and other data is still being explored. There are many possibilities being considered such as the changing recharge rate of the pixel and other QIE effects. However, these effects are less understood and are harder to implement in the simulation. Once these flaws are corrected, the SiPM simulation will be a more useful tool beyond this analysis. 

Because the tests on the new readout modules showed they functioned as expect, they were installed in the HE on the CMS detector during the winter of 2018. Similar readout modules, also containing SiPMs, are being constructed for the HB. While the test beam did answer many questions about the SiPM there are still features, such as nonlinearity, that could be further explored. Overall the tests and analyses on the new readout modules were very successful, and the information obtained will be used to examine the data from the CMS detector.

