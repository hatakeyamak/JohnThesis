\section{Analysis Results}

The tests that were run on these new readout modules were the final tests before installation on the HE on the CMS detector. With the LHC now producing high energy collisions at a faster rate, these new electronics will be very helpful in finding new particles. The readout modules in the test beam performed as expected and showed they are capable of performing the task for which they were designed. We were also able to extract useful information about the readout modules. The pulse shape analysis using the test beam data was very successful. It was expected that the pulse shape would be similar to a Landau Gaussian function which is what the analysis showed. It also gave consistent results over different runs and showed little change at increased energies. This pulse shape will allow the extraction of individual particle signals in the CMS detector. 

While the non-linearity of the SiPM was another reason for the test beam, it was not able to be studied in depth. The SiPM is approximately linear at low incident photo electrons. To study the SiPM non-linearity, a particle at an energy that will produce a number of photo electrons in the non-linear range, is needed. After looking at the data it was found that the test beam is not capable of producing particles at the necessary energy. The test beam did show an energy range for which the SiPM is near linear.

The SiPM simulation was constructed to supplement the other analyzes and illustrate different features of the SiPM in a more intuitive manner. When the results of the simulation were compared to results from other analyzes, however, it showed that there were still some flaws in the construction of the simulation. The two main things I studied in my work, the SiPM non-linearity and the pulse shape, both showed significant deviation when the simulation was compared to physical data. 

\section{Final Thoughts}

The reason for the discrepancy between the simulation and other data is still being explored. There are many possibilities being considered such as the changing recharge rate of the pixel and other QIE effects. However, these effects are less understood and are harder to implement in the simulation. Once these flaws are corrected, the SiPM simulation will be a useful tool beyond this analysis. 

Because the tests on the new readout modules showed they functioned as expect, they were installed in the HE on the CMS detector during the winter of 2018. Similar readout modules, also containing SiPMs, are being constructed for the HB. While the test beam did answer many questions about the SiPM there are still features, like non-linearity, that could be further explored. Overall the tests and analyzes on the new readout modules were very successfuly, and the information obtained will be used to examine the data from the CMS detector.

